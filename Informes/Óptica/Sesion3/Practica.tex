\documentclass[10pt,onecolumn]{article}


%Aquí vienen todos los paquetes que se van a usar

\usepackage[spanish]{babel}
\usepackage[letterpaper,top=2cm,bottom=2cm,left=3cm,right=3cm,marginparwidth=1.75cm]{geometry}

\usepackage[bottom]{footmisc}
\usepackage{comment}
\usepackage{subcaption} % Añade esto en el preámbulo si no lo tienes
\usepackage{colortbl} %Permite agregar color a las celdas de las tablas.
\usepackage{mathtools} %Contiene herramientas adicionales para componer fórmulas matemáticas en LaTeX.
\usepackage{algorithm2e} %Facilita la sintaxis de los algoritmos.
\usepackage[T1]{fontenc} %Hace posible escribir en diferentes idiomas.
\usepackage{lmodern} %Carga la fuente Latin modern en el documento.
\usepackage{microtype} %Mejoras tipográficas (interletrado y protrusión) para una composición más estética.
\usepackage{float} %Hace posible el uso de [H] en las tablas y gráficas.
\usepackage{blindtext} %Necesario para usar este tipo de comentarios.
\usepackage{multicol} %Posibilidad (no mostrada aquí) de escribir en varias columnas para una misma página.
\usepackage{wrapfig} %Permite que las figuras y tablas se ajusten mejor al texto.
\usepackage{breqn} %Ajusta ecuaciones largas al ancho de la página, teniendo que romper y bajar a una fila más abajo parte de la ecuación para que quepa dentro de los márgenes.
\usepackage{array} %Permite personalizar más ampliamente las columnas de las tablas de LaTeX.
\usepackage{booktabs} %Tablas de alta calidad (\toprule, \midrule, \bottomrule) sin líneas verticales.
\usepackage{eurosym} %Posibilidad de usar el símbolo € en el texto.
\usepackage{amsmath} %Mejoras para la escritura de ecuaciones matemáticas.
\usepackage{mathrsfs} %Proporciona el alfabeto de letras minúsculas de Ralph Smith.
\usepackage{indentfirst} %Sangra el primer párrafo de un encabezado del documento, si por ejemplo el primer párrafo queremos que esté sangrado.
\usepackage{amsthm} %Facilita la escritura de demostraciones de afirmaciones matemáticas de toda clase.
\usepackage[font=small,labelfont=bf]{caption} %Formatea los títulos de figuras y gráficas.
\usepackage{graphicx} %Facilita la inclusión de imágenes en LaTeX.
\usepackage[titles]{tocloft} %Posibilidad de personalizar el formato del Índice.
\usepackage[hidelinks]{hyperref} %Crea hipervínculos en el PDF generado del documento de texto final, para poder navegar a través y externamente al documento.
\usepackage{bookmark} %Evita el mensaje de rerunfilecheck sobre Practica.out
\setlength{\parindent}{12pt} %Establece la sangría de la primera línea de cada párrafo a {...pt}.
\usepackage{multirow} %Posibilidad de combinar filas de las tablas de LaTeX.
\usepackage{parskip} %Modifica los espacios entre párrafos para que no haya sangría, sino que se genere un espacio entre medias predeterminado. 
\usepackage{fancyhdr} %Creación de encabezados y pies de página.
\usepackage[utf8]{inputenc} %Permite usar caracteres especiales y letras acentuadas directamente en el código LaTeX.
\usepackage{bm} %Posibilidad de escribir símbolos en negrita en lenguaje matemático. 
\usepackage{url} %Inclusión de direcciones URL en el documento.
\usepackage{siunitx} %Permite la escritura de cantidades físicas y unidades en el documento.
% Configuración de siunitx para notación española (coma decimal) y detección de estilos
% Configuración explícita de separador decimal en español
\sisetup{output-decimal-marker = {,}, input-decimal-markers = {,}, detect-weight=true, detect-family=true}
%\usepackage{geometry} %Posibilidad de personalizar las dimensiones del documento.
% Eliminado paquete geometry duplicado (ya cargado arriba con opciones)
% Eliminado fontenc duplicado (ya cargado arriba)



%Aquí se definen cosas de colores y movidas fancys
\usepackage[dvipsnames]{xcolor}
\definecolor{nube}{RGB}{188,108,37}
\definecolor{salmon}{RGB}{248, 131, 121}

%Aquí tenemos un poco el formateo de la página
\pagestyle{fancy}
\fancyhf{}
\cfoot{}
\rhead{\thepage}
 \lhead{Práctica 5: Microscopio. Anillos de Newton}


\begin{document}
\renewcommand{\headrulewidth}{0.5pt}
\newcommand{\HRule}[1]{\rule{\linewidth}{#1}}
\renewcommand{\refname}{Bibliografía}
\renewcommand{\tablename}{Tabla}
\renewcommand{\contentsname}{Índice}
\renewcommand{\figurename}{Figura}
\captionsetup{labelfont={color=nube}}
\addtocontents{toc}{\hspace{-7.5mm} \textbf{Capítulos}}
\addtocontents{toc}{\hfill \textbf{Página} \par}
\addtocontents{toc}{\vspace{-2mm} \hspace{-7.5mm} \hrule \par}

\onecolumn
\begin{titlepage}
\centering
    {\HRule{2 pt}} \\
    \vspace{0.5cm}
    {\scshape\Huge {\textbf{Práctica 5: Microscopio. Anillos de Newton}  }}  \\
    %AQUI SE PONE EL TITULO
    \vspace{1 mm}
    
    {\HRule{2 pt}}

    \vspace{1cm}
    \Large 5/11/2025 \\
    \vspace{1cm}
 
    \normalfont\Large Universidad de Granada, Facultad de Ciencias \\
    \vspace{0.5cm}
    
    
     \normalfont\Large Grado en Físicas \\
    \vspace{0.5cm}
    \normalfont Óptica I\\%
    \vspace{1.5cm}

\centering
    {\includegraphics[width=0.5\textwidth]{UGR-MARCA-01-color.jpg}\par}

\vfill
    \vspace{1cm}
    \scshape\Large Jorge del Rio López \\
    \scshape\Large Paula Roca Gómez\\

    \vspace{0.5cm}
    \scshape\Large P5

\vfill

\end{titlepage}

\selectlanguage{spanish}

\tableofcontents %único comando para manejar TODO el índice.
\newpage

\HRule{0.5pt} %Controla los márgenes horizontales visibles (líneas) y el grosor de la línea (...pt).

\begin{abstract}
\end{abstract}

\section{Resultados}

\subsection{Microscopio}

\subsection{Anillos de Newton}

Ahora pasamos a la segunda parte de la práctica, en la que se nos pide medir los radios de los anillos de Newton. Para ello, utilizamos un microscopio para observar los anillos y medir sus radios correspondientes a diferentes órdenes.

En la tabla \ref{tab:r1_r2} se muestran los valores medidos en el laboratorio de los radios de los anillos de Newton $r_1$ y $r_2$ correspondientes a los órdenes $k=1$ y $k=7$, respectivamente. Cada medida se realizó tres veces para obtener un valor más preciso, y la incertidumbre asociada a cada medida es de $u_C(r_1) = 0{,}02$\,mm y $u_C(r_2) = 0{,}02$\,mm.

\begin{table}[H]
\centering
\begin{tabular}{|c|c|}
\hline
\rowcolor[rgb]{ .651,  .788,  .925}
$r_{1}$(mm) & $r_{2}$(mm) \\ \hline
\rowcolor[rgb]{.816,  .816,  .816} 1,16 & 3,87 \\ \hline
\rowcolor[rgb]{.816,  .816,  .816} 1,17 & 3,83 \\ \hline
\rowcolor[rgb]{.816,  .816,  .816} 1,19 & 3,59 \\ \hline
\end{tabular}
\caption{Valores medidos en el laboratorio de los radios de los anillos de Newton $r_1$ y $r_2$ correspondientes a los órdenes $k=1$ y $k=7$, respectivamente.La incertidumbre asociada a cada medida es de $u_C(r_1) = 0{,}02$\,mm y $u_C(r_2) = 0{,}02$\,mm.}
\label{tab:r1_r2}
\end{table}


Una vez obtuvimos los promedios de los datos medidos con sus incertidumbres, procedimos a calcular el radio de curvatura $R$ de la lente esférica mediante la fórmula proporcionada en el informe de prácticas\cite{InfoOpticaPrisma}:
\begin{equation}
    R = \dfrac{r^2_{k1}-r^2_{k2}}{\lambda\ (k_1 - k_2)}
\end{equation}
Donde $r_{k1}$ y $r_{k2}$ son los radios medios de los anillos de Newton correspondientes a los órdenes $k_1$ y $k_2$, respectivamente, y $\lambda$ es la longitud de onda de la luz utilizada (en este caso, la luz amarilla del sodio con $\lambda = 589{,}3$\,nm). Los valores de $k_1$ y $k_2$ son los órdenes de los anillos medidos, en nuestro caso $k_1 = 1$ y $k_2 = 7$. 

Sustituyendo los valores obtenidos en la fórmula, calculamos $R$, obteniendo así el siguiente resultado, $R_1=3600\pm	100$ (mm).

Tambien se nos pedia obtener el valor de $R$ mediante otro método,este era un ajuste por minimos cuadrados de los datos obtenidos. Para ello, representamos gráficamente $r^2$ frente a $k\lambda$ y realizamos un ajuste lineal de los datos. La pendiente de la recta ajustada nos permite calcular el radio de curvatura $R$ mediante la relación:
\begin{equation}
    R = \dfrac{1}{\text{pendiente}}
\end{equation}

Nosotros decidimos realizar 5 ordenes de anillos, desde el $k=1$ hasta el $k=5$ y tambien medimos los radios de los anillos correspondientes tres veces para obtener un valor más preciso, llegando así a los siguientes resultados:

\begin{table}[H]
\centering
\begin{tabular}{|c|c|}
\hline
\rowcolor[rgb]{ .651,  .788,  .925}
$r^2$ (mm)& $u_C(r^2)$ (mm) \\ \hline
\rowcolor[rgb]{.816,  .816,  .816} 1,42 & 0,06 \\ \hline
\rowcolor[rgb]{.816,  .816,  .816} 2,3 & 0,1 \\ \hline
\rowcolor[rgb]{.816,  .816,  .816} 4,1 & 0,2 \\ \hline
\rowcolor[rgb]{.816,  .816,  .816} 6,4 & 0,3 \\ \hline
\rowcolor[rgb]{.816,  .816,  .816} 11 & 1 \\ \hline
\end{tabular}
\caption{Valores medidos en el laboratorio de los radios de los anillos de Newton $r$ correspondientes a los órdenes $k=1$ y $k=5$.La incertidumbre asociada a cada medida ha sido calculado mediante tipo A y tipo B.}
\label{tab:r2_ur}
\end{table}

Esto nos permite realizar el ajuste lineal de los datos, obteniendo la siguiente gráfica:

\begin{figure}[H]
    \centering
    \includegraphics[width=0.7\textwidth]{AnillosNewton.png}
    \caption{Ajuste lineal de los datos obtenidos en la práctica de los anillos de Newton. La pendiente de la recta ajustada nos permite calcular el radio de curvatura $R$ de la lente esférica.}
    \label{fig:AnillosNewton}
\end{figure}

A partir del ajuste lineal, obtenemos la pendiente de la recta ajustada, que es $m = 0,00028745\pm 0,00000004$ (mm)$^{-1}$. Utilizando esta pendiente, calculamos el radio de curvatura $R$ mediante la relación mencionada anteriormente, obteniendo así el siguiente resultado, $R_2=3470\pm50$ (mm).

\section{Conclusiones}

En esta práctica hemos aprendido a utilizar un microscopio compuesto para observar objetos pequeños y medir sus dimensiones con precisión. Hemos calibrado el microscopio utilizando una micrométrica y hemos medido el tamaño de varias muestras, obteniendo resultados consistentes con las dimensiones conocidas de los objetos. Además, hemos estudiado los anillos de Newton formados por la interferencia de la luz reflejada entre una lente esférica y una superficie plana. Hemos medido los radios de los anillos correspondientes a diferentes órdenes y hemos calculado el radio de curvatura de la lente utilizando dos métodos distintos: mediante la fórmula directa y mediante un ajuste lineal de los datos obtenidos. Ambos métodos nos han proporcionado resultados semejantes, lo que confirma la validez de nuestras mediciones y cálculos.


\section{Agradecimientos}
Nos gustaría agradecer al Departamento de Óptica de la Universidad de Granada por proporcionarnos
los medios y el material necesarios para llevar a cabo esta práctica, así como a nuestro profesor por su guía y apoyo durante el desarrollo de la misma.


\newpage
\section{Apéndices}
\subsection{A1: Cálculo de incertidumbres} 
\subsubsection{Sensibilidad de los instrumentos}
En los cálculos se ha considerado la sensibilidad (resolución) de los instrumentos empleados:
\begin{itemize}
    \item Micrómetro para $s$ y $d'_{1,2}$: $\delta = 0{,}01$\,mm (incertidumbre tipo B asociada $u_B = \delta/\sqrt{12} \approx 0{,}003$\,mm).
    \item Distancia focal $f'$ del sistema: se ha tratado como dato sin incertidumbre
\end{itemize}
\subsubsection{Cálculo de la desviación estándar}
La desviación estándar se utilizará para el cálculo de la incertidumbre tipo A; su expresión es:
\begin{equation}\centering
    s = \sqrt{\dfrac{1}{N - 1} \sum_{i=1}^{N} (x_i - \bar{x})^2}
\end{equation}

\subsubsection{Incertidumbre tipo A}
La incertidumbre tipo A se evalúa mediante análisis estadístico de datos repetidos, basada en su dispersión o desviación estándar, para lograr dar un valor que se asemeje lo máximo posible al real; su expresión es la siguiente:
\begin{equation}\centering
u_A = \dfrac{s}{\sqrt{n}}
\end{equation}
donde $s$ es la desviación estándar y $n$ el número de medidas realizadas.

\subsubsection{Incertidumbre tipo B}
La incertidumbre tipo B se debe al error que ocasiona medir con instrumentos inexactos; se calcula a partir de la resolución ($\delta$) del instrumento utilizado:
\begin{equation}\centering
    u_B = \dfrac{\delta}{\sqrt{12}}
\end{equation}

\subsubsection{Incertidumbre Combinada}
Tras obtener la incertidumbre tipo A y la tipo B, debemos combinarlas para dar un valor concreto de incertidumbre; se calcula de la siguiente forma:

\begin{equation}
    u_C = \sqrt{(u_A)^2\ +\ (u_B)^2}
\end{equation}

\subsubsection{Incertidumbre debida a medida indirecta}
Dado que usaremos este tipo de incertidumbre en varias ocasiones, la dejaremos aquí definida para evitar 
tener que repetir el proceso cada vez. Sea $f(x_1,x_2,\ldots,x_n)$ una función con $n$ variables; la incertidumbre
de esta función se calcula mediante la propagación de incertidumbres, y obtenemos la siguiente expresión:

\begin{equation}
    \boxed{u_C(f(x_1,x_2,\ldots,x_n)) = \sqrt{\sum_{i=1}^{n} \left(\dfrac{\partial f}{\partial x_i}\right)^2\ u_C(x_i)^2}}
\end{equation}

Donde $\dfrac{\partial f}{\partial x_i}$ es la derivada parcial de $f$ respecto a la variable $x_i$, y $u_C(x_i)$ es la incertidumbre combinada de la variable $x_i$.
Este cálculo nos proporciona la incertidumbre de una función que depende de varias variables, teniendo en cuenta las incertidumbres individuales de cada variable.
\subsubsection{Incertidumbre por cambio de unidades}
Esta incertidumbre la emplearemos cuando un valor se encuentre en unidades distintas a las del SI. 
Se calcula mediante propagación de incertidumbres, considerando la función 
$f(x) = \dfrac{x}{K}$, con $K$ una constante real. Entonces, $u_C(f(x)) = \sqrt{\left(\dfrac{\partial f(x)}{\partial x}\right)^2 u_C(x)^2}$, 
y como $\dfrac{\partial f(x)}{\partial x} = \dfrac{1}{K}$, finalmente obtenemos:

\begin{equation}
    \boxed{u_C(f(x)) = \dfrac{u_C(x)}{K}}
\end{equation}

\subsubsection{Incertidumbre de $R$}
La incertidumbre de $R$ se calcula mediante la propagación de incertidumbres, considerando la función $R = \dfrac{r^2_{k1}-r^2_{k2}}{\lambda\ (k_1 - k_2)}$, donde $\lambda$ es una constante sin incertidumbre. Por tanto, aplicando la fórmula de propagación de incertidumbres, obtenemos:


\begin{equation}
    \boxed{u_C(R) = \dfrac{1}{\lambda(k_1-k_2)}\sqrt{((2 r_{k1}- r_{k2}^2)u_C(r_{k1}))^2 + ((r_{k1}^2-2 r_{k2})u_C(r_{k2}))^2}}
\end{equation}

% Bibliografía

\bibliographystyle{plain} % o el estilo que uses
\bibliography{biblio,references} % sin extensión, apunta a biblio.bib

\end{document}

\documentclass[10pt,onecolumn]{article}


%Aquí vienen todos los paquetes que se van a usar

\usepackage[spanish]{babel}
\usepackage[letterpaper,top=2cm,bottom=2cm,left=3cm,right=3cm,marginparwidth=1.75cm]{geometry}

\usepackage[bottom]{footmisc}
\usepackage{comment}
\usepackage{subcaption} % Añade esto en el preámbulo si no lo tienes
\usepackage{colortbl} %Permite agregar color a las celdas de las tablas.
\usepackage{mathtools} %Contiene herramientas adicionales para componer fórmulas matemáticas en LaTeX.
\usepackage{algorithm2e} %Facilita la sintaxis de los algoritmos.
\usepackage[T1]{fontenc} %Hace posible escribir en diferentes idiomas.
\usepackage{lmodern} %Carga la fuente Latin modern en el documento.
\usepackage{microtype} %Mejoras tipográficas (interletrado y protrusión) para una composición más estética.
\usepackage{float} %Hace posible el uso de [H] en las tablas y gráficas.
\usepackage{blindtext} %Necesario para usar este tipo de comentarios.
\usepackage{multicol} %Posibilidad (no mostrada aquí) de escribir en varias columnas para una misma página.
\usepackage{wrapfig} %Permite que las figuras y tablas se ajusten mejor al texto.
\usepackage{breqn} %Ajusta ecuaciones largas al ancho de la página, teniendo que romper y bajar a una fila más abajo parte de la ecuación para que quepa dentro de los márgenes.
\usepackage{array} %Permite personalizar más ampliamente las columnas de las tablas de LaTeX.
\usepackage{booktabs} %Tablas de alta calidad (\toprule, \midrule, \bottomrule) sin líneas verticales.
\usepackage{eurosym} %Posibilidad de usar el símbolo € en el texto.
\usepackage{amsmath} %Mejoras para la escritura de ecuaciones matemáticas.
\usepackage{mathrsfs} %Proporciona el alfabeto de letras minúsculas de Ralph Smith.
\usepackage{indentfirst} %Sangra el primer párrafo de un encabezado del documento, si por ejemplo el primer párrafo queremos que esté sangrado.
\usepackage{amsthm} %Facilita la escritura de demostraciones de afirmaciones matemáticas de toda clase.
\usepackage[font=small,labelfont=bf]{caption} %Formatea los títulos de figuras y gráficas.
\usepackage{graphicx} %Facilita la inclusión de imágenes en LaTeX.
\usepackage[titles]{tocloft} %Posibilidad de personalizar el formato del Índice.
\usepackage[hidelinks]{hyperref} %Crea hipervínculos en el PDF generado del documento de texto final, para poder navegar a través y externamente al documento.
\usepackage{bookmark} %Evita el mensaje de rerunfilecheck sobre Practica.out
\setlength{\parindent}{12pt} %Establece la sangría de la primera línea de cada párrafo a {...pt}.
\usepackage{multirow} %Posibilidad de combinar filas de las tablas de LaTeX.
\usepackage{parskip} %Modifica los espacios entre párrafos para que no haya sangría, sino que se genere un espacio entre medias predeterminado. 
\usepackage{fancyhdr} %Creación de encabezados y pies de página.
\usepackage[utf8]{inputenc} %Permite usar caracteres especiales y letras acentuadas directamente en el código LaTeX.
\usepackage{bm} %Posibilidad de escribir símbolos en negrita en lenguaje matemático. 
\usepackage{url} %Inclusión de direcciones URL en el documento.
\usepackage{siunitx} %Permite la escritura de cantidades físicas y unidades en el documento.
% Configuración de siunitx para notación española (coma decimal) y detección de estilos
% Configuración explícita de separador decimal en español
\sisetup{output-decimal-marker = {,}, input-decimal-markers = {,}, detect-weight=true, detect-family=true}
%\usepackage{geometry} %Posibilidad de personalizar las dimensiones del documento.
% Eliminado paquete geometry duplicado (ya cargado arriba con opciones)
% Eliminado fontenc duplicado (ya cargado arriba)



%Aquí se definen cosas de colores y movidas fancys
\usepackage[dvipsnames]{xcolor}
\definecolor{nube}{RGB}{188,108,37}
\definecolor{salmon}{RGB}{248, 131, 121}

%Aquí tenemos un poco el formateo de la página
\pagestyle{fancy}
\fancyhf{}
\cfoot{}
\rhead{\thepage}
 \lhead{Práctica 2: Estudio del Prisma}


\begin{document}
\renewcommand{\headrulewidth}{0.5pt}
\newcommand{\HRule}[1]{\rule{\linewidth}{#1}}
\renewcommand{\refname}{Bibliografía}
\renewcommand{\tablename}{Tabla}
\renewcommand{\contentsname}{Índice}
\renewcommand{\figurename}{Figura}
\captionsetup{labelfont={color=nube}}
\addtocontents{toc}{\hspace{-7.5mm} \textbf{Capítulos}}
\addtocontents{toc}{\hfill \textbf{Página} \par}
\addtocontents{toc}{\vspace{-2mm} \hspace{-7.5mm} \hrule \par}

\onecolumn
\begin{titlepage}
\centering
    {\HRule{2 pt}} \\
    \vspace{0.5cm}
    {\scshape\Huge {\textbf{Práctica 2: Estudio del Prisma}  }}  \\
    %AQUI SE PONE EL TITULO
    \vspace{1 mm}
    
    {\HRule{2 pt}}

    \vspace{1cm}
    \Large 22/10/2025 \\
    \vspace{1cm}
 
    \normalfont\Large Universidad de Granada, Facultad de Ciencias \\
    \vspace{0.5cm}
    
    
     \normalfont\Large Grado en Físicas \\
    \vspace{0.5cm}
    \normalfont Óptica I\\%
    \vspace{1.5cm}

\centering
    {\includegraphics[width=0.5\textwidth]{UGR-MARCA-01-color.jpg}\par}

\vfill
    \vspace{1cm}
    \scshape\Large Jorge del Rio López \\
    \scshape\Large Paula Roca Gómez\\

    \vspace{0.5cm}
    \scshape\Large P5

\vfill

\end{titlepage}

\selectlanguage{spanish}

\tableofcontents %único comando para manejar TODO el índice.
\newpage

\HRule{0.5pt} %Controla los márgenes horizontales visibles (líneas) y el grosor de la línea (...pt).

\begin{abstract}
Con el espectrogoniómetro medimos la desviación mínima de varias líneas espectrales y, a partir de las lecturas L y L0, obtuvimos los índices de refracción del prisma. Con esos valores calculamos su poder dispersivo y el número de Abbe correspondiente.
\end{abstract}

PUTA MAAAAAAA

\section{Resultados y Discusiones}
\subsection{Determinación de $\delta$}
En esta práctica hemos medido el ángulo de desviación mínima para tres longitudes de onda distintas, para ello
comenzamos midiendo la posición angular $L_0$ a la que llegaría la luz si no hubiera prisma,
y la posición angular $L$ a la que llega la luz tras pasar por el prisma. Obteniendo así los siguientes valores para las distintas longitudes de onda:

Para la banda amarilla:

\begin{table}[H]
\centering
\begin{tabular}{|r|r|r|r|}
\hline
\rowcolor[rgb]{ .651,  .788,  .925}
\multicolumn{1}{|l|}{$L$ (º)} & \multicolumn{1}{l|}{$u_C(L)$ (º)} & \multicolumn{1}{l|}{$L_0$ (º)} & \multicolumn{1}{l|}{$u_C(L_0)$ (º)} \\ \hline
\rowcolor[rgb]{.816,  .816,  .816} 34,067 & 0,017 & 74,000 & 0,017 \\ \hline
\rowcolor[rgb]{.816,  .816,  .816} 33,417 & 0,017 & 73,983 & 0,017 \\ \hline
\rowcolor[rgb]{.816,  .816,  .816} 34,050 & 0,017 & 74,000 & 0,017 \\ \hline
\end{tabular}
\caption{Valores obtenidos en el espectrogoniómetro para la lámpara de sodio. Esta nos proporcionaba la banda amarilla.}\label{tab:banda_amarilla}
\end{table}


Para la banda roja:
\begin{table}[htbp]
\centering
\begin{tabular}{|r|r|r|r|}
\hline
\rowcolor[rgb]{ .651,  .788,  .925}
\multicolumn{1}{|l|}{$L$ (º)} & \multicolumn{1}{l|}{$u_C(L)$ (º)} & \multicolumn{1}{l|}{$L_0$ (º)} & \multicolumn{1}{l|}{$u_C(L_0)$ (º)} \\ \hline
\rowcolor[rgb]{.816,  .816,  .816}220,967 & 0,017 & 260,717 & 0,017 \\ \hline
\rowcolor[rgb]{.816,  .816,  .816}220,050 & 0,017 & 260,583 & 0,017 \\ \hline
\rowcolor[rgb]{.816,  .816,  .816}220,733 & 0,017 & 260,650 & 0,017 \\ \hline
\end{tabular}
\caption{Valores obtenidos en el espectrogoniómetro para la lámpara de helio. Esta nos proporcionaba la banda roja.}\label{tab:banda_roja}
\end{table}

Para la banda azul:
\begin{table}[H]
\centering
\begin{tabular}{|r|r|r|r|}
\hline
\rowcolor[rgb]{ .651,  .788,  .925}
\multicolumn{1}{|l|}{$L$ (º)} & \multicolumn{1}{l|}{$u_C(L)$ (º)} & \multicolumn{1}{l|}{$L_0$ (º)} & \multicolumn{1}{l|}{$u_C(L_0)$ (º)} \\ \hline
\rowcolor[rgb]{.816,  .816,  .816} 219,983 & 0,017 & 260,717 & 0,017 \\ \hline
\rowcolor[rgb]{.816,  .816,  .816} 219,083 & 0,017 & 260,583 & 0,017 \\ \hline
\rowcolor[rgb]{.816,  .816,  .816} 219,617 & 0,017 & 260,650 & 0,017 \\ \hline
\end{tabular}
\caption{Valores obtenidos en el espectrogoniómetro para la lámpara de helio. Esta nos proporcionaba la banda azul.}\label{tab:banda_azul}
\end{table}

A continuación, empleamos la expresión proporcionada por el informe \cite{InfoOpticaPrisma}, $\delta = |L - L_0|$, para calcular el ángulo de desviación mínima para cada longitud de onda,esta angulo lo usaremos más adelante para obtener los valores requeridos.Exponemos los resultados:\footnote{Los valores del amarillo pasarán a llamarse $D$, los del rojo $C$ y los del azul $F$.}
\begin{table}[H]
\centering
\begin{tabular}{|r|r|r|}
\hline
\rowcolor[rgb]{ .651,  .788,  .925}
\multicolumn{1}{|l|}{$\delta_C$ (º)} & \multicolumn{1}{l|}{$\delta_D$ (º)} & \multicolumn{1}{l|}{$\delta_F$ (º)} \\ \hline
\rowcolor[rgb]{.816,  .816,  .816}  39,7500 & 39,9333 & 40,7333 \\ \hline
\rowcolor[rgb]{.816,  .816,  .816} 40,5333  & 40,5667 & 41,5000 \\ \hline
\rowcolor[rgb]{.816,  .816,  .816} 39,9167& 39,9500  & 41,0333 \\ \hline
\end{tabular}
\caption{Valores obtenidos del ángulo de desviación mínima para cada longitud de onda. Todos los valores en grados y con incertidumbre de 0,0068 (º).}\label{tab:delta}
\end{table}


Observamos que se cumple la relación esperada entre ellas, $(\delta_m)_F\ \text{mayor que}\ (\delta_m)_D\ \text{mayor que}\ (\delta_m)_C$.
Pero estas medidas están en grados, por lo tanto hay que pasarlas a radianes para poder usarlas en la fórmula del índice de refracción, obteniendo los siguientes valores:

\begin{table}[H]
\centering
\begin{tabular}{|r|r|r|}
\hline
\rowcolor[rgb]{ .651,  .788,  .925}
\multicolumn{1}{|l|}{$\delta_D$ (rad)} & \multicolumn{1}{l|}{$\delta_C$ (rad)} & \multicolumn{1}{l|}{$\delta_F$ (rad)} \\ \hline
\rowcolor[rgb]{.816,  .816,  .816}0,69377 & 0,69697 & 0,71093 \\ \hline
\rowcolor[rgb]{.816,  .816,  .816}0,70744 & 0,70802 & 0,72431 \\ \hline
\rowcolor[rgb]{.816,  .816,  .816}0,69668 & 0,69726 & 0,71617 \\ \hline
\end{tabular}
\caption{Valores obtenidos del ángulo de desviación mínima para cada longitud de onda. Todos los valores en radianes y con incertidumbre de 0,00012(rad)}\label{tab:delta_rad}
\end{table}

\subsection{Determinación del indice de refracción, número de Abbe y poder Dispersivo}

Con los datos del apartado anterior,  sabiendo que el ángulo de refringencia del prisma tratado es de 60º sin incertidumbre podemos calcular el índice de refracción del prisma para cada longitud de onda. Para elloutilizamos la siguiente expresión: 
\[
n = \frac{\sen\left( \frac{\delta_m + \alpha}{2} \right)}{\sen\left( \frac{\alpha}{2} \right)}
\]
La siguiente tabla expone los valores obtenidos del indice de refracción para cada longitud de onda:

\begin{table}[H]
\centering
\begin{tabular}{|r|r|r|r|r|r|}
\hline
\rowcolor[rgb]{ .651,  .788,  .925}
\multicolumn{1}{|l|}{$n_f$} & \multicolumn{1}{l|}{$u_C(n_f)$} & \multicolumn{1}{l|}{$n_c$} & \multicolumn{1}{l|}{$u_C(n_c)$} & \multicolumn{1}{l|}{$n_d$} & \multicolumn{1}{l|}{$u_C(n_d)$} \\ \hline
\rowcolor[rgb]{.816,  .816,  .816}1,540285 & 0,000076 & 1,531341 & 0,000076 & 1,529281 & 0,000077 \\ \hline
\rowcolor[rgb]{.816,  .816,  .816}1,548785 & 0,000075 & 1,538427 & 0,000076 & 1,538056 & 0,000076 \\ \hline
\rowcolor[rgb]{.816,  .816,  .816}1,543619 & 0,000076 & 1,531528 & 0,000076 & 1,531154 & 0,000076 \\ \hline
\end{tabular}
\caption{Valores obtenidos del índice de refracción para cada longitud de onda.}\label{tab:indice_refraccion}
\end{table}

Tras obtener estos valores podemos calcular el número de Abbe del prisma, usando la expresión proporcionada en el informe~\cite{InfoOpticaPrisma}:
 \[
\nu_D = \frac{n_D - 1}{n_F - n_C}
\]
Para poder utilizar esta ecuación haremos un promedio\footnote{Debido a que la incertidumbre considerada es practicamente la misma no haremos una media ponderada, sino un media normal.} de los valores obtenidos para cada 
índice de refracción, obteniendo los siguientes valores medios:

$n_F = 1,5442\pm 0,0020$	

$n_C = 1,5338\pm 0,0019$	

$n_D = 1,5328\pm 0,0022$

Obtenemos $\nu_D=46\pm12$\footnote{El proceso de cálculo de la incertidumbre se explica en el apéndice}. 
Dado que para los vidrios ópticos convencionales el valor se encuentra entre 25 y 75,
el número de Abbe obtenido de forma experimental 
se encuentra en el rango aceptable. Con el valor obtenido, sería un vidrio de tipo flint, ya que $\nu_D\ \text{menor que}\ 50$.
Sin embargo, la incertidumbre obtenida es considerable, tanto que no podemos determinar con seguridad si
se trata de un vidrio de tipo flint o de tipo crown.
La magnitud de la incertidumbre puede deberse a que no fue sencillo determinar en qué posición el sistema 
se encontraba en desviación mínima, debido a factores como la exactitud del calibrado del anteojo en ese 
momento, o el propio error humano.


Podemos calcular el poder dispersivo del vidrio, definido como $\frac{1}{\nu_D} = \frac{n_F - n_C}{n_D - 1}$. 
Obteniendo $\frac{1}{\nu_D}=0.0214\pm 0.0056$. De nuevo, la incertidumbre es alta como para 
sacar conclusiones sobre las propiedades del vidrio estudiado.

\section{Conclusiones}
En esta práctica hemos profundizado en la propiedad dispersiva de los prismas, con el objetivo 
final de calcular el índice de refracción del prisma para tres longitudes de onda distintas, 
y con ello calcular el número de Abbe y el poder dispersivo del prisma.

El método utilizado ha sido la medición del ángulo de desviación mínimo de la luz, 
utilizando la comparación entre la posición a la que llegaría la luz de no haber prisma, 
y la posición a la que llega tras pasar por él.

Se obtuvo la relación de ángulo de desviación mínima esperada entre las distintas longitudes de onda, hecho que nos da constancia de que la práctica se ha realizado de forma correcta aunque los datos tomados creíamos que estaban mal.

Con estos datos, se ha calculado el índice de refracción para cada caso, y con estos valores se ha calculado 
el número de Abbe. El resultado en esta ocasión no ha sido del todo satisfactorio, ya que aunque el 
valor obtenido es razonable, la incertidumbre asociada a él hace que no sea una experiencia correcta 
para determinar las características dispersivas del prisma estudiado. Esto mismo se aplica al valor del 
poder dispersivo del prisma.

\section{Agradecimientos}
Nos gustaría agradecer al Departamento de Óptica de la Universidad de Granada por proporcionarnos
los medios y el material necesario para llevar a cabo esta práctica, así como a 
nuestro profesor por su guía y apoyo durante el desarrollo de la misma.
También nos gustaría agradecer a HyperPhysics por sus explicaciones sobre los cristales ópticos proporcionadas
en~\cite{CristalesOpticos}.

\newpage
\section{Apéndices}
\subsection{A1: Cálculo de incertidumbres} 
\subsubsection{Cálculo de la desviación estándar}
La desviación estándar se usará para el cálculo de la incertidumbre tipo A; su ecuación sería:
\begin{equation}\centering
    s = \sqrt{\dfrac{1}{N - 1} \sum_{i=1}^{N} (x_i - \bar{x})}
\end{equation}

\subsubsection{Incertidumbre tipo A}
La incertidumbre tipo A se evalúa mediante análisis estadístico de datos repetidos, basada en su dispersión o desviación estándar, para lograr dar un valor que se asemeje lo máximo posible al real; su ecuación es la siguiente:
\begin{equation}\centering
u_A = \dfrac{s}{\sqrt{n}}
\end{equation}
donde $s$ sería la desviación estándar y $n$ el número de medidas realizadas.

\subsubsection{Incertidumbre tipo B}
La incertidumbre tipo B se debe al error que ocasiona el medir con instrumentos inexactos, su ecuación involucra la resolución ($\delta$) del instrumento que se haya usado:
\begin{equation}\centering
    u_B = \dfrac{\delta}{\sqrt{12}}
\end{equation}

\subsubsection{Incertidumbre Combinada}
Tras obtener la incertidumbre tipo A y tipo B debemos juntarlas para dar un valor de incertidumbre concreto, se calcularía de la siguiente forma:
\begin{equation}
    u_C = \sqrt{(u_A)^2\ +\ (u_B)^2}
\end{equation}

\subsubsection{Incertidumbre debida a medida indirecta}
Debido a la que usaremos este tipo de incertidumbre en varias ocasiones, la dejaremos aquí definida para evitar 
tener que repetir el proceso cada vez. Sea $f(x_1,x_2,\ldots,x_n)$ una funcion con $n$ variables, la incertidumbre
de esta función se calcularía mediante la propagación de incertidumbres, y obtendríamos la siguiente ecuación:

\begin{equation}
    \boxed{u_C(f(x_1,x_2,\ldots,x_n)) = \sqrt{\sum_{i=1}^{n} \left(\dfrac{\partial f}{\partial x_i}\right)^2\ u_C(x_i)^2}}
\end{equation}

Donde $\dfrac{\partial f}{\partial x_i}$ es la derivada parcial de $f$ respecto a la variable $x_i$, y $u_C(x_i)$ es la incertidumbre combinada de la variable $x_i$.
Este cálculo nos proporciona la incertidumbre de una función que depende de varias variables, teniendo en cuenta las incertidumbres individuales de cada variable.
\subsubsection{Incertidumbre por cambio de unidades}
Esta incertidumbre la usaremos cuando un valor se encuentre en unas unidades distintas a las del SI, 
se llevará a cabo mediante la propagación de incertidumbres, suponiendo  la siguiente ecuación: 
$f(x) = \dfrac{x}{K}$ siendo $k$ un valor cualquiera real, entonces, $u_C(f(x)) = \sqrt{(\dfrac{\partial f(x)}{\partial x})^2\ u_C(x)^2}$ 
y como $\dfrac{\partial f(x)}{\partial x} = \dfrac{1}{K}$, finalmente obtenemos:

\begin{equation}
    \boxed{u_C(f(x)) = \dfrac{u_C(x)}{K}}
\end{equation}

\subsubsection{Incertidumbre de la suma}
Supongamos que tenemos la ecuación $f(x)=x_1+x_2$, la incertidumbre de esta suma se calcularía con la propagación de incertidumbre, y obtendríamos la siguiente ecuación:

\begin{equation}
    \boxed{u_C(f(x))=\sqrt{u_C(x_1)^2+u_C(x_2)^2}}
\end{equation}

\subsubsection{Incertidumbre de la inversa}
Sea $f(x) = \dfrac{1}{x}$, haciendo el proceso de propagación de incertidumbres obtenemos su incertidumbre como:

\begin{equation}
    \boxed{u_C(f(x)) = \dfrac{u_C(x)}{x^2}}
\end{equation}

La incertidumbre del poder dispersivo vendría dada por esta fórmula.

\subsubsection{Incertidumbre del índice de refracción}
Sea $n = \dfrac{\sen(\dfrac{\alpha + \delta_m}{2})}{\sen(\dfrac{\alpha}{2})}$, haciendo el proceso de propagación de incertidumbres 
y sabiendo que 
$\dfrac{\partial n}{\partial \delta_m} = \dfrac{1}{2}\ \dfrac{\cos(\dfrac{\alpha + \delta_m}{2})}{\sen(\dfrac{\alpha}{2})}$ obtenemos su incertidumbre como:

\begin{equation}
    \boxed{u_C(n) = \dfrac{1}{2}\ \dfrac{\cos(\dfrac{\alpha + \delta_m}{2})}{\sen(\dfrac{\alpha}{2})} u_C(\delta_m)}
\end{equation}

\subsubsection{Incertidumbre del número de Abbe}
Sea $\nu_D = \dfrac{n_D - 1}{n_F - n_C}$, haciendo el proceso de propagación de incertidumbres 
y sabiendo que: 
$\dfrac{\partial \nu_D}{\partial n_F} = \dfrac{n_D - 1}{(n_F - n_C)^2}$, $\dfrac{\partial \nu_D}{\partial n_C} = \dfrac{n_D - 1}{(n_F - n_C)^2}$ y $\dfrac{\partial \nu_D}{\partial n_D} = -\dfrac{1}{n_F - n_C}$ obtenemos su incertidumbre como:

\begin{equation}
    \boxed{u_C(\nu_D) =\dfrac{n_D - 1}{n_F - n_C} \sqrt{\left(\dfrac{u_C(n_F)}{n_F - n_C} \right)^2 + \left(\dfrac{u_C(n_C)}{n_F - n_C} \right)^2 + \left(\dfrac{u_C(n_D)}{(n_D - 1)^2} \right)^2}}
\end{equation}

% Bibliografía

\bibliographystyle{plain} % o el estilo que uses
\bibliography{biblio,references} % sin extensión, apunta a biblio.bib

\end{document}

\documentclass[10pt,onecolumn]{article}


%Aquí vienen todos los paquetes que se van a usar

\usepackage[spanish]{babel}
\usepackage[letterpaper,top=2cm,bottom=2cm,left=3cm,right=3cm,marginparwidth=1.75cm]{geometry}

\usepackage[bottom]{footmisc}
\usepackage{comment}
\usepackage{subcaption} % Añade esto en el preámbulo si no lo tienes
\usepackage{colortbl} %Permite agregar color a las celdas de las tablas.
\usepackage{mathtools} %Contiene herramientas adicionales para componer fórmulas matemáticas en LaTeX.
\usepackage{algorithm2e} %Facilita la sintaxis de los algoritmos.
\usepackage[T1]{fontenc} %Hace posible escribir en diferentes idiomas.
\usepackage{lmodern} %Carga la fuente Latin modern en el documento.
\usepackage{microtype} %Mejoras tipográficas (interletrado y protrusión) para una composición más estética.
\usepackage{float} %Hace posible el uso de [H] en las tablas y gráficas.
\usepackage{blindtext} %Necesario para usar este tipo de comentarios.
\usepackage{multicol} %Posibilidad (no mostrada aquí) de escribir en varias columnas para una misma página.
\usepackage{wrapfig} %Permite que las figuras y tablas se ajusten mejor al texto.
\usepackage{breqn} %Ajusta ecuaciones largas al ancho de la página, teniendo que romper y bajar a una fila más abajo parte de la ecuación para que quepa dentro de los márgenes.
\usepackage{array} %Permite personalizar más ampliamente las columnas de las tablas de LaTeX.
\usepackage{booktabs} %Tablas de alta calidad (\toprule, \midrule, \bottomrule) sin líneas verticales.
\usepackage{eurosym} %Posibilidad de usar el símbolo € en el texto.
\usepackage{amsmath} %Mejoras para la escritura de ecuaciones matemáticas.
\usepackage{mathrsfs} %Proporciona el alfabeto de letras minúsculas de Ralph Smith.
\usepackage{indentfirst} %Sangra el primer párrafo de un encabezado del documento, si por ejemplo el primer párrafo queremos que esté sangrado.
\usepackage{amsthm} %Facilita la escritura de demostraciones de afirmaciones matemáticas de toda clase.
\usepackage[font=small,labelfont=bf]{caption} %Formatea los títulos de figuras y gráficas.
\usepackage{graphicx} %Facilita la inclusión de imágenes en LaTeX.
\usepackage[titles]{tocloft} %Posibilidad de personalizar el formato del Índice.
\usepackage[hidelinks]{hyperref} %Crea hipervínculos en el PDF generado del documento de texto final, para poder navegar a través y externamente al documento.
\usepackage{bookmark} %Evita el mensaje de rerunfilecheck sobre Practica.out
\setlength{\parindent}{12pt} %Establece la sangría de la primera línea de cada párrafo a {...pt}.
\usepackage{multirow} %Posibilidad de combinar filas de las tablas de LaTeX.
\usepackage{parskip} %Modifica los espacios entre párrafos para que no haya sangría, sino que se genere un espacio entre medias predeterminado. 
\usepackage{fancyhdr} %Creación de encabezados y pies de página.
\usepackage[utf8]{inputenc} %Permite usar caracteres especiales y letras acentuadas directamente en el código LaTeX.
\usepackage{bm} %Posibilidad de escribir símbolos en negrita en lenguaje matemático. 
\usepackage{url} %Inclusión de direcciones URL en el documento.
\usepackage{siunitx} %Permite la escritura de cantidades físicas y unidades en el documento.
% Configuración de siunitx para notación española (coma decimal) y detección de estilos
% Configuración explícita de separador decimal en español
\sisetup{output-decimal-marker = {,}, input-decimal-markers = {,}, detect-weight=true, detect-family=true}
%\usepackage{geometry} %Posibilidad de personalizar las dimensiones del documento.
% Eliminado paquete geometry duplicado (ya cargado arriba con opciones)
% Eliminado fontenc duplicado (ya cargado arriba)



%Aquí se definen cosas de colores y movidas fancys
\usepackage[dvipsnames]{xcolor}
\definecolor{nube}{RGB}{188,108,37}
\definecolor{salmon}{RGB}{248, 131, 121}

%Aquí tenemos un poco el formateo de la página
\pagestyle{fancy}
\fancyhf{}
\cfoot{}
\rhead{\thepage}
 \lhead{Práctica 4: Biprisma de Fresnel}


\begin{document}
\renewcommand{\headrulewidth}{0.5pt}
\newcommand{\HRule}[1]{\rule{\linewidth}{#1}}
\renewcommand{\refname}{Bibliografía}
\renewcommand{\tablename}{Tabla}
\renewcommand{\contentsname}{Índice}
\renewcommand{\figurename}{Figura}
\captionsetup{labelfont={color=nube}}
\addtocontents{toc}{\hspace{-7.5mm} \textbf{Capítulos}}
\addtocontents{toc}{\hfill \textbf{Página} \par}
\addtocontents{toc}{\vspace{-2mm} \hspace{-7.5mm} \hrule \par}

\onecolumn
\begin{titlepage}
\centering
    {\HRule{2 pt}} \\
    \vspace{0.5cm}
    {\scshape\Huge {\textbf{Práctica 4: Biprisma de Fresnel}  }}  \\
    %AQUI SE PONE EL TITULO
    \vspace{1 mm}
    
    {\HRule{2 pt}}

    \vspace{1cm}
    \Large 29/10/2025 \\
    \vspace{1cm}
 
    \normalfont\Large Universidad de Granada, Facultad de Ciencias \\
    \vspace{0.5cm}
    
    
     \normalfont\Large Grado en Físicas \\
    \vspace{0.5cm}
    \normalfont Óptica I\\%
    \vspace{1.5cm}

\centering
    {\includegraphics[width=0.5\textwidth]{UGR-MARCA-01-color.jpg}\par}

\vfill
    \vspace{1cm}
    \scshape\Large Jorge del Rio López \\
    \scshape\Large Paula Roca Gómez\\

    \vspace{0.5cm}
    \scshape\Large P5

\vfill

\end{titlepage}

\selectlanguage{spanish}

\tableofcontents %único comando para manejar TODO el índice.
\newpage

\HRule{0.5pt} %Controla los márgenes horizontales visibles (líneas) y el grosor de la línea (...pt).

\begin{abstract}
En la práctica del biprisma de Fresnel, se generaron y analizaron franjas de interferencia utilizando una fuente puntual de luz y el biprisma para obtener dos focos coherentes. Se midió la distancia entre franjas y se estudió el contraste de la interferencia según las condiciones de coherencia espacial para poder obtener el valor de la longitud de onda con la que la lámpara emitía.
\end{abstract}

\section{Resultados}
Tras configurar el sistema óptico según el esquema propuesto en el guión de prácticas~\cite{InfoOpticaPrisma},
comenzamos midiendo la separación entre tres franjas de la fuente, obteniendo así los valores de $s = x_{k+3} - x_k$, donde $x_i$ es la posición de la $i$-ésima franja medida con el micrómetro.

Los valores medidos se muestran en la siguiente tabla:
\begin{table}[H]
\centering
\begin{tabular}{|c|}
\hline
\rowcolor[rgb]{ .651,  .788,  .925}
\textbf{s (mm)} \\ \hline
\rowcolor[rgb]{.816,  .816,  .816} 3,020 \\ \hline
\rowcolor[rgb]{.816,  .816,  .816} 3,010 \\ \hline
\rowcolor[rgb]{.816,  .816,  .816} 3,070 \\ \hline
\end{tabular}
\caption{Medidas de \texorpdfstring{$s$}{s} (mm), con incertidumbre \texorpdfstring{$u_s$}{u\_s} = 0,01\,mm}\label{tab:medidas_S}
\end{table}

Con estos datos, obtenemos el valor promedio de $s = 3,03\,\pm\,0,02$\,mm. La desviación estándar de las tres medidas es $\sigma \approx 0,032$\,mm, por lo que consideramos que tres medidas son suficientes para este caso.

Posteriormente, introducimos en el sistema una lente, y buscamos las posiciones de Bessel, es decir, aquellas en las que observamos por el ocular dos líneas bien definidas.
Anotamos las distancias entre rendijas en cada una de las dos posiciones de Bessel estudiadas, $d'_1$ y $d'_2$, de las cuales $d'_1$ corresponde a la primera posición. En la siguiente tabla se muestran los valores medidos.
\begin{table}[H]
\centering
\begin{tabular}{|c|c|}
\hline
\rowcolor[rgb]{ .651,  .788,  .925}
\textbf{Medidas posición Bessel $d'_1$ (mm)} & \textbf{Medidas posición Bessel $d'_2$ (mm)} \\ \hline
\rowcolor[rgb]{.816,  .816,  .816} -0,35 & -3,54 \\ \hline
\rowcolor[rgb]{.816,  .816,  .816} -0,34 & -3,55 \\ \hline
\rowcolor[rgb]{.816,  .816,  .816} -0,40 & -3,55 \\ \hline
\end{tabular}
\caption{Medidas de las interfranjas obtenidas para las posiciones de Bessel: \texorpdfstring{$d'_1$}{d'\_1} (mm) y \texorpdfstring{$d'_2$}{d'\_2} (mm). La incertidumbre en ambos casos es \texorpdfstring{$u_{d'}$}{u\_{d'}} = 0,01\,mm}
\label{tab:franjas_cerca_lejos}
\end{table}

A partir de estos valores, podemos promediarlos para obtener $d'_2 = -3,55\,\pm\,0,01$\,mm y $d'_1 = -0,36\,\pm\,0,02$\,mm. Las dispersiones muestrales son $\sigma(d'_1) \approx 0,032$\,mm y $\sigma(d'_2) \approx 0,006$\,mm, suficientemente pequeñas como para justificar el uso de tres medidas.

Con estos datos, podemos calcular la distancia $d$ como:
\begin{equation}
d = \sqrt{d'_{1} d'_{2}}
\end{equation}
y hallamos $d = 1,14\,\pm\,0,03$\,mm.

Con los datos de las interfranjas en las posiciones de Bessel y sabiendo que la distancia focal del biprisma utilizado es de $f' = 15$ (cm), calculamos la distancia de las fuentes al lugar de interferencia, $D$. 
Esta variable guarda la siguiente relación con las demás:

\begin{equation}
D = f' \left( 2 - \frac{d^{2} + d'_{1}{}^{2}}{d\, d'_{1}} \right)
\end{equation}

Tras operar con estos valores, se obtiene $D = 817\,\pm\,12$\,mm.

En este momento podemos calcular la longitud de onda de la fuente utilizada mediante la expresión
\begin{equation}
\lambda = \frac{d \, s}{3D}
\end{equation}

Con los valores experimentales medidos, encontramos que la longitud de onda calculada es $\lambda = 1410\,\pm\,40$\,nm.\footnote{La incertidumbre del valor de $\lambda$ se compone del cálculo para determinarla en mm y de su posterior conversión a nm.}

El valor teórico de la longitud de onda para la lámpara de sodio utilizada es de 589,3 (nm) \cite{InfoOpticaPrisma}, por lo que el valor obtenido experimentalmente no concuerda con el valor teórico esperado, lo que nos da un error relativo de $\epsilon_R = 138\%$.

El error puede deberse a varios factores, entre ellos la dificultad de medir con precisión las posiciones de las franjas y las distancias entre ellas, así como posibles errores en la alineación del sistema óptico. Además, la calidad del biprisma y la coherencia de la fuente de luz también pueden influir en los resultados obtenidos.


\section{Conclusiones}
Aunque finalmente no hemos conseguido un valor de la longitud de onda cercano al teórico, hemos podido observar y analizar el fenómeno de interferencia utilizando el biprisma de Fresnel. La práctica nos ha permitido comprender mejor los conceptos de coherencia espacial y la formación de franjas de interferencia, así como la importancia de la precisión en las mediciones experimentales.



\section{Agradecimientos}
Nos gustaría agradecer al Departamento de Óptica de la Universidad de Granada por proporcionarnos
los medios y el material necesarios para llevar a cabo esta práctica, así como a nuestro profesor por su guía y apoyo durante el desarrollo de la misma.
\newpage
\section{Apéndices}
\subsection{A1: Cálculo de incertidumbres} 
\subsubsection{Sensibilidad de los instrumentos}
En los cálculos se ha considerado la sensibilidad (resolución) de los instrumentos empleados:
\begin{itemize}
    \item Micrómetro para $s$ y $d'_{1,2}$: $\delta = 0{,}01$\,mm (incertidumbre tipo B asociada $u_B = \delta/\sqrt{12} \approx 0{,}003$\,mm).
    \item Distancia focal $f'$ del sistema: se ha tratado como dato sin incertidumbre
\end{itemize}
\subsubsection{Cálculo de la desviación estándar}
La desviación estándar se utilizará para el cálculo de la incertidumbre tipo A; su expresión es:
\begin{equation}\centering
    s = \sqrt{\dfrac{1}{N - 1} \sum_{i=1}^{N} (x_i - \bar{x})^2}
\end{equation}

\subsubsection{Incertidumbre tipo A}
La incertidumbre tipo A se evalúa mediante análisis estadístico de datos repetidos, basada en su dispersión o desviación estándar, para lograr dar un valor que se asemeje lo máximo posible al real; su expresión es la siguiente:
\begin{equation}\centering
u_A = \dfrac{s}{\sqrt{n}}
\end{equation}
donde $s$ es la desviación estándar y $n$ el número de medidas realizadas.

\subsubsection{Incertidumbre tipo B}
La incertidumbre tipo B se debe al error que ocasiona medir con instrumentos inexactos; se calcula a partir de la resolución ($\delta$) del instrumento utilizado:
\begin{equation}\centering
    u_B = \dfrac{\delta}{\sqrt{12}}
\end{equation}

\subsubsection{Incertidumbre Combinada}
Tras obtener la incertidumbre tipo A y la tipo B, debemos combinarlas para dar un valor concreto de incertidumbre; se calcula de la siguiente forma:

\begin{equation}
    u_C = \sqrt{(u_A)^2\ +\ (u_B)^2}
\end{equation}

\subsubsection{Incertidumbre debida a medida indirecta}
Dado que usaremos este tipo de incertidumbre en varias ocasiones, la dejaremos aquí definida para evitar 
tener que repetir el proceso cada vez. Sea $f(x_1,x_2,\ldots,x_n)$ una función con $n$ variables; la incertidumbre
de esta función se calcula mediante la propagación de incertidumbres, y obtenemos la siguiente expresión:

\begin{equation}
    \boxed{u_C(f(x_1,x_2,\ldots,x_n)) = \sqrt{\sum_{i=1}^{n} \left(\dfrac{\partial f}{\partial x_i}\right)^2\ u_C(x_i)^2}}
\end{equation}

Donde $\dfrac{\partial f}{\partial x_i}$ es la derivada parcial de $f$ respecto a la variable $x_i$, y $u_C(x_i)$ es la incertidumbre combinada de la variable $x_i$.
Este cálculo nos proporciona la incertidumbre de una función que depende de varias variables, teniendo en cuenta las incertidumbres individuales de cada variable.
\subsubsection{Incertidumbre por cambio de unidades}
Esta incertidumbre la emplearemos cuando un valor se encuentre en unidades distintas a las del SI. 
Se calcula mediante propagación de incertidumbres, considerando la función 
$f(x) = \dfrac{x}{K}$, con $K$ una constante real. Entonces, $u_C(f(x)) = \sqrt{\left(\dfrac{\partial f(x)}{\partial x}\right)^2 u_C(x)^2}$, 
y como $\dfrac{\partial f(x)}{\partial x} = \dfrac{1}{K}$, finalmente obtenemos:

\begin{equation}
    \boxed{u_C(f(x)) = \dfrac{u_C(x)}{K}}
\end{equation}

\subsubsection{Incertidumbre de $d$}
Para calcular la incertidumbre de $d = \sqrt{d'_{1} d'_{2}}$, aplicamos la fórmula de propagación de incertidumbres, y calculamos las derivadas parciales correspondientes, obteniendo:
\begin{equation}
    \boxed{u_C(d) = \dfrac{1}{2\ d}\sqrt{(d_1\ u_C(d_1))^2\ +\ (d_2\ u_C(d_2))^2} }
\end{equation}


\subsubsection{Incertidumbre de $\lambda$}
Para calcular la incertidumbre de $\lambda = \dfrac{d\ s}{3D}$, aplicamos la fórmula de propagación de incertidumbres, y calculamos las derivadas parciales correspondientes, obteniendo:
\begin{equation}
    \boxed{u_C(\lambda) = \dfrac{sd}{3D}\sqrt{(\dfrac{u_C(s)}{s})^2 + (\dfrac{u_C(d)}{d})^2 + (\dfrac{u_C(D)}{D})^2}}
\end{equation}

\subsubsection{Incertidumbre de $D$}
Para calcular la incertidumbre de $D = f' \left( 2 - \dfrac{d^{2} + d'_{1}{}^{2}}{d\, d'_{1}} \right)$, aplicamos la fórmula de propagación de incertidumbres, y calculamos las derivadas parciales correspondientes, obteniendo:
\begin{equation}
    \boxed{u_C(D)= f'\ \sqrt{(u_C(d))^2\ (\dfrac{d_1'}{d^2} - \dfrac{1}{d_1'})^2 + (u_C(d_1'))^2\ (\dfrac{d}{d_1^2} - \dfrac{1}{d})^2}}
\end{equation}
% Bibliografía

\bibliographystyle{plain} % o el estilo que uses
\bibliography{biblio,references} % sin extensión, apunta a biblio.bib

\end{document}

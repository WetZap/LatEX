\documentclass[10pt,onecolumn]{article}


%Aquí vienen todos los paquetes que se van a usar

\usepackage[spanish]{babel}
\usepackage[letterpaper,top=2cm,bottom=2cm,left=3cm,right=3cm,marginparwidth=1.75cm]{geometry}

\usepackage[bottom]{footmisc}
\usepackage{comment}
\usepackage{subcaption} % Añade esto en el preámbulo si no lo tienes
\usepackage{colortbl} %Permite agregar color a las celdas de las tablas.
\usepackage{mathtools} %Contiene herramientas adicionales para componer fórmulas matemáticas en LaTeX.
\usepackage{algorithm2e} %Facilita la sintaxis de los algoritmos.
\usepackage[T1]{fontenc} %Hace posible escribir en diferentes idiomas.
\usepackage{lmodern} %Carga la fuente Latin modern en el documento.
\usepackage{microtype} %Mejoras tipográficas (interletrado y protrusión) para una composición más estética.
\usepackage{float} %Hace posible el uso de [H] en las tablas y gráficas.
\usepackage{blindtext} %Necesario para usar este tipo de comentarios.
\usepackage{multicol} %Posibilidad (no mostrada aquí) de escribir en varias columnas para una misma página.
\usepackage{wrapfig} %Permite que las figuras y tablas se ajusten mejor al texto.
\usepackage{breqn} %Ajusta ecuaciones largas al ancho de la página, teniendo que romper y bajar a una fila más abajo parte de la ecuación para que quepa dentro de los márgenes.
\usepackage{array} %Permite personalizar más ampliamente las columnas de las tablas de LaTeX.
\usepackage{booktabs} %Tablas de alta calidad (\toprule, \midrule, \bottomrule) sin líneas verticales.
\usepackage{eurosym} %Posibilidad de usar el símbolo € en el texto.
\usepackage{amsmath} %Mejoras para la escritura de ecuaciones matemáticas.
\usepackage{mathrsfs} %Proporciona el alfabeto de letras minúsculas de Ralph Smith.
\usepackage{indentfirst} %Sangra el primer párrafo de un encabezado del documento, si por ejemplo el primer párrafo queremos que esté sangrado.
\usepackage{amsthm} %Facilita la escritura de demostraciones de afirmaciones matemáticas de toda clase.
\usepackage[font=small,labelfont=bf]{caption} %Formatea los títulos de figuras y gráficas.
\usepackage{graphicx} %Facilita la inclusión de imágenes en LaTeX.
\usepackage[titles]{tocloft} %Posibilidad de personalizar el formato del Índice.
\usepackage[hidelinks]{hyperref} %Crea hipervínculos en el PDF generado del documento de texto final, para poder navegar a través y externamente al documento.
\usepackage{bookmark} %Evita el mensaje de rerunfilecheck sobre Practica.out
\setlength{\parindent}{12pt} %Establece la sangría de la primera línea de cada párrafo a {...pt}.
\usepackage{multirow} %Posibilidad de combinar filas de las tablas de LaTeX.
\usepackage{parskip} %Modifica los espacios entre párrafos para que no haya sangría, sino que se genere un espacio entre medias predeterminado. 
\usepackage{fancyhdr} %Creación de encabezados y pies de página.
\usepackage[utf8]{inputenc} %Permite usar caracteres especiales y letras acentuadas directamente en el código LaTeX.
\usepackage{bm} %Posibilidad de escribir símbolos en negrita en lenguaje matemático. 
\usepackage{url} %Inclusión de direcciones URL en el documento.
\usepackage{siunitx} %Permite la escritura de cantidades físicas y unidades en el documento.
% Configuración de siunitx para notación española (coma decimal) y detección de estilos
% Configuración explícita de separador decimal en español
\sisetup{output-decimal-marker = {,}, input-decimal-markers = {,}, detect-weight=true, detect-family=true}
%\usepackage{geometry} %Posibilidad de personalizar las dimensiones del documento.
% Eliminado paquete geometry duplicado (ya cargado arriba con opciones)
% Eliminado fontenc duplicado (ya cargado arriba)



%Aquí se definen cosas de colores y movidas fancys
\usepackage[dvipsnames]{xcolor}
\definecolor{nube}{RGB}{188,108,37}
\definecolor{salmon}{RGB}{248, 131, 121}

%Aquí tenemos un poco el formateo de la página
\pagestyle{fancy}
\fancyhf{}
\cfoot{}
\rhead{\thepage}
 \lhead{Práctica 4: Biprisma de Fresnel}


\begin{document}
\renewcommand{\headrulewidth}{0.5pt}
\newcommand{\HRule}[1]{\rule{\linewidth}{#1}}
\renewcommand{\refname}{Bibliografía}
\renewcommand{\tablename}{Tabla}
\renewcommand{\contentsname}{Índice}
\renewcommand{\figurename}{Figura}
\captionsetup{labelfont={color=nube}}
\addtocontents{toc}{\hspace{-7.5mm} \textbf{Capítulos}}
\addtocontents{toc}{\hfill \textbf{Página} \par}
\addtocontents{toc}{\vspace{-2mm} \hspace{-7.5mm} \hrule \par}

\onecolumn
\begin{titlepage}
\centering
    {\HRule{2 pt}} \\
    \vspace{0.5cm}
    {\scshape\Huge {\textbf{Práctica 4: Biprisma de Fresnel}  }}  \\
    %AQUI SE PONE EL TITULO
    \vspace{1 mm}
    
    {\HRule{2 pt}}

    \vspace{1cm}
    \Large 29/10/2025 \\
    \vspace{1cm}
 
    \normalfont\Large Universidad de Granada, Facultad de Ciencias \\
    \vspace{0.5cm}
    
    
     \normalfont\Large Grado en Físicas \\
    \vspace{0.5cm}
    \normalfont Óptica I\\%
    \vspace{1.5cm}

\centering
    {\includegraphics[width=0.5\textwidth]{UGR-MARCA-01-color.jpg}\par}

\vfill
    \vspace{1cm}
    \scshape\Large Jorge del Rio López \\
    \scshape\Large Paula Roca Gómez\\

    \vspace{0.5cm}
    \scshape\Large P5

\vfill

\end{titlepage}

\selectlanguage{spanish}

\tableofcontents %único comando para manejar TODO el índice.
\newpage

\HRule{0.5pt} %Controla los márgenes horizontales visibles (líneas) y el grosor de la línea (...pt).

\begin{abstract}

\end{abstract}

\section{Resultados}
Comenzamos midiendo la separación entre tres franjas de la fuente, $s$, obteniendo:
\begin{table}[H]
\centering
\begin{tabular}{|c|}
\hline
\rowcolor[rgb]{ .651,  .788,  .925}
\textbf{s (mm)} \\ \hline
\rowcolor[rgb]{.816,  .816,  .816} 3,02 \\ \hline
\rowcolor[rgb]{.816,  .816,  .816} 3,01 \\ \hline
\rowcolor[rgb]{.816,  .816,  .816} 3,07 \\ \hline
\end{tabular}
\caption{Medidas de s (mm), con incertidumbre $u_s=0,01$ (mm)}
\label{tab:medidas_S}
\end{table}

obtenemos el valor promedio de $s=3,033\pm 0,021$ (mm).

Posteriormente, introducimos en el sistema el biprisma de Fresnel, y buscamos las posiciones de Bessel, es decir, aquellas en las que observamos por el ocular dos líneas bien definidas.
Anotamos las distancias entre rendijas en cada una de las dos posiciones de Bessel estudiadas, $d'_1$ y $d'_2$, de las cuáles $d1'$ corresponde a la posición (A CUÁL!!!!!!!). Se muestran en la siguiente tabla los valores medidos.
\begin{table}[H]
\centering
\begin{tabular}{|c|c|}
\hline
\rowcolor[rgb]{ .651,  .788,  .925}
\textbf{Medidas posición Bessel $d'_1$ (mm)} & \textbf{Medidas posición Bessel $d'_2$ (mm)} \\ \hline
\rowcolor[rgb]{.816,  .816,  .816} 3,54 & 0,35 \\ \hline
\rowcolor[rgb]{.816,  .816,  .816} 3,55 & 0,34 \\ \hline
\rowcolor[rgb]{.816,  .816,  .816} 3,55 & 0,40 \\ \hline
\end{tabular}
\caption{Medidas de las interfranjas obtenidas para las posiciones de Bessel, $d'_1$ (mm) y ($d'_2$)(mm). La incertidumbre en ambos casos es $u_d'=0,01$ (mm)}
\label{tab:franjas_cerca_lejos}
\end{table}

Lo que deriva en unos valores promedio de $d'_1=3,5467\pm 0,0033$(mm) y $d'_2=0,363\pm 0,021$ (mm) .

Con los datos de las interfranjas en las posiciones de Bessel y sabiendo que la distancia focal del biprisma utilizado es de $f'=15$ (cm), calculamos la distancia de las fuentes al lugar de interferencia, $D$. 
Sabiendo la relación entre las variables:
\begin{equation}
D = f' \left( 2 - \frac{d^{2} + d'_{1}{}^{2}}{d\, d'_{1}} \right)
\end{equation}

se obtiene $D=816,7 \pm 4,4$ (mm).

Por otro lado, calculamos la distancia $d$ entre las fuentes como 
\begin{equation}
d = \sqrt{d'_{1} d'_{2}}
\end{equation}
y hallamos $d=1,135 \pm 0,033$ (mm).

En este momento podemos calcular la longitud de onda de la fuente utilizada mediante la expresión
\begin{equation}
\lambda = \frac{d \, s}{3D}
\end{equation}

Con los valores experimentales medidos encontramos que la longitud de onda calculada es $\lambda=1405,465449 nm$ INCERTIDUMBRE.





\section{Conclusiones}


\section{Agradecimientos}

\newpage
\section{Apéndices}
\subsection{A1: Cálculo de incertidumbres} 
\subsubsection{Cálculo de la desviación estándar}
La desviación estándar se usará para el cálculo de la incertidumbre tipo A; su ecuación sería:
\begin{equation}\centering
    s = \sqrt{\dfrac{1}{N - 1} \sum_{i=1}^{N} (x_i - \bar{x})}
\end{equation}

\subsubsection{Incertidumbre tipo A}
La incertidumbre tipo A se evalúa mediante análisis estadístico de datos repetidos, basada en su dispersión o desviación estándar, para lograr dar un valor que se asemeje lo máximo posible al real; su ecuación es la siguiente:
\begin{equation}\centering
u_A = \dfrac{s}{\sqrt{n}}
\end{equation}
donde $s$ sería la desviación estándar y $n$ el número de medidas realizadas.

\subsubsection{Incertidumbre tipo B}
La incertidumbre tipo B se debe al error que ocasiona el medir con instrumentos inexactos, su ecuación involucra la resolución ($\delta$) del instrumento que se haya usado:
\begin{equation}\centering
    u_B = \dfrac{\delta}{\sqrt{12}}
\end{equation}

\subsubsection{Incertidumbre Combinada}
Tras obtener la incertidumbre tipo A y tipo B debemos juntarlas para dar un valor de incertidumbre concreto, se calcularía de la siguiente forma:
\begin{equation}
    u_C = \sqrt{(u_A)^2\ +\ (u_B)^2}
\end{equation}

\subsubsection{Incertidumbre debida a medida indirecta}
Debido a la que usaremos este tipo de incertidumbre en varias ocasiones, la dejaremos aquí definida para evitar 
tener que repetir el proceso cada vez. Sea $f(x_1,x_2,\ldots,x_n)$ una funcion con $n$ variables, la incertidumbre
de esta función se calcularía mediante la propagación de incertidumbres, y obtendríamos la siguiente ecuación:

\begin{equation}
    \boxed{u_C(f(x_1,x_2,\ldots,x_n)) = \sqrt{\sum_{i=1}^{n} \left(\dfrac{\partial f}{\partial x_i}\right)^2\ u_C(x_i)^2}}
\end{equation}

Donde $\dfrac{\partial f}{\partial x_i}$ es la derivada parcial de $f$ respecto a la variable $x_i$, y $u_C(x_i)$ es la incertidumbre combinada de la variable $x_i$.
Este cálculo nos proporciona la incertidumbre de una función que depende de varias variables, teniendo en cuenta las incertidumbres individuales de cada variable.
\subsubsection{Incertidumbre por cambio de unidades}
Esta incertidumbre la usaremos cuando un valor se encuentre en unas unidades distintas a las del SI, 
se llevará a cabo mediante la propagación de incertidumbres, suponiendo  la siguiente ecuación: 
$f(x) = \dfrac{x}{K}$ siendo $k$ un valor cualquiera real, entonces, $u_C(f(x)) = \sqrt{(\dfrac{\partial f(x)}{\partial x})^2\ u_C(x)^2}$ 
y como $\dfrac{\partial f(x)}{\partial x} = \dfrac{1}{K}$, finalmente obtenemos:

\begin{equation}
    \boxed{u_C(f(x)) = \dfrac{u_C(x)}{K}}
\end{equation}

\subsubsection{Incertidumbre de la suma}
Supongamos que tenemos la ecuación $f(x)=x_1+x_2$, la incertidumbre de esta suma se calcularía con la propagación de incertidumbre, y obtendríamos la siguiente ecuación:

\begin{equation}
    \boxed{u_C(f(x))=\sqrt{u_C(x_1)^2+u_C(x_2)^2}}
\end{equation}

\subsubsection{Incertidumbre de la inversa}
Sea $f(x) = \dfrac{1}{x}$, haciendo el proceso de propagación de incertidumbres obtenemos su incertidumbre como:

\begin{equation}
    \boxed{u_C(f(x)) = \dfrac{u_C(x)}{x^2}}
\end{equation}

La incertidumbre del poder dispersivo vendría dada por esta fórmula.

\subsubsection{Incertidumbre del índice de refracción}
Sea $n = \dfrac{\sen(\dfrac{\alpha + \delta_m}{2})}{\sen(\dfrac{\alpha}{2})}$, haciendo el proceso de propagación de incertidumbres 
y sabiendo que 
$\dfrac{\partial n}{\partial \delta_m} = \dfrac{1}{2}\ \dfrac{\cos(\dfrac{\alpha + \delta_m}{2})}{\sen(\dfrac{\alpha}{2})}$ obtenemos su incertidumbre como:

\begin{equation}
    \boxed{u_C(n) = \dfrac{1}{2}\ \dfrac{\cos(\dfrac{\alpha + \delta_m}{2})}{\sen(\dfrac{\alpha}{2})} u_C(\delta_m)}
\end{equation}

\subsubsection{Incertidumbre del número de Abbe}
Sea $\nu_D = \dfrac{n_D - 1}{n_F - n_C}$, haciendo el proceso de propagación de incertidumbres 
y sabiendo que: 
$\dfrac{\partial \nu_D}{\partial n_F} = \dfrac{n_D - 1}{(n_F - n_C)^2}$, $\dfrac{\partial \nu_D}{\partial n_C} = \dfrac{n_D - 1}{(n_F - n_C)^2}$ y $\dfrac{\partial \nu_D}{\partial n_D} = -\dfrac{1}{n_F - n_C}$ obtenemos su incertidumbre como:

\begin{equation}
    \boxed{u_C(\nu_D) =\dfrac{n_D - 1}{n_F - n_C} \sqrt{\left(\dfrac{u_C(n_F)}{n_F - n_C} \right)^2 + \left(\dfrac{u_C(n_C)}{n_F - n_C} \right)^2 + \left(\dfrac{u_C(n_D)}{(n_D - 1)^2} \right)^2}}
\end{equation}

% Bibliografía

\bibliographystyle{plain} % o el estilo que uses
\bibliography{biblio,references} % sin extensión, apunta a biblio.bib

\end{document}
